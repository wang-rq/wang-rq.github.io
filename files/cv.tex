%-------------------------
% Entry-level Resume in LaTeX
% Last Edits - July 19, 2021
% Author : Jayesh Sanwal
% Reach out to me on LinkedIn(/in/jsanwal), with any suggestions, ideas, etc.
%------------------------

%%-------------------------------------
% Author Notes here -
% 1) Change " Author+an = {N=highlight}" in "Publications.bib", where N is the number at which your name appears
% 2) Use "\vspace" wisely, that would change spacing, and is currently being used as a hacky fix
% 3) Do not delete the fonts in the left side
% 4) Use the "Rich Text" feature on Overleaf - On the top panel next to source. Makes it much easier for starters on LaTeX to use this template
% 5) Do NOT use periods at the end of bullet points, the sample (ipsum) text might have it
% 6) Use "maxbibnames" on this file to change the maximum number of authors on the paper (Credits: Dr. Natasha Krell) - Default is 3, but change line 92 to add authors

%%-------------------------------------
% Changes from last version -
% 1) Removed areas of Expertise as a separate section - integrated into the summary section
% 2) Changed Focus to Academic Projects instead of Education
% 3) Made the integrated summary 3 lines over 1
% 4) Changed spacing between sections and bullets
% 5) Changed the number of authors to be user customized

%%-------------------------------------
% Changes to be made here -
% 1) Shift to Class file, make changes here
% 2) Use a fonts folder
% 3) Incorporate Leadership & volunteering together
% 4) Remove \vspace based 'hacky' fixes
%%-------------------------------------


\documentclass[a4,11pt]{article}

%%%%%%% --------------------------------------------------------------------------------------
%%%%%%%  STARTING HERE, DO NOT TOUCH ANYTHING 
%%%%%%% --------------------------------------------------------------------------------------

\usepackage{latexsym}
\usepackage[empty]{fullpage}
\usepackage{titlesec}
 \usepackage{marvosym}
\usepackage[usenames,dvipsnames]{color}
\usepackage{verbatim}
\usepackage[hidelinks]{hyperref}
\usepackage{fancyhdr}
\usepackage{multicol}
\usepackage{hyperref}
\usepackage{csquotes}
\usepackage{tabularx}
\hypersetup{colorlinks=true,urlcolor=black}
\usepackage[11pt]{moresize}
\usepackage{setspace}
\usepackage{fontspec}
\usepackage[inline]{enumitem}
\usepackage{array}
\usepackage{hyperref}
\hypersetup{
    colorlinks=true,
    linkcolor=blue,
    filecolor=blue,      
    urlcolor=blue,
    citecolor=cyan,
}
\newcolumntype{P}[1]{>{\centering\arraybackslash}p{#1}}
\usepackage{anyfontsize}

%%%% Set Margins
\usepackage[margin=1.2cm, top=1.2cm]{geometry}

%%%% Set Fonts
\setmainfont[
BoldFont=SourceSansPro-Semibold.otf, %SourceSansPro-Bold.otf
ItalicFont=SourceSansPro-RegularIt.otf
]{SourceSansPro-Regular.otf}
\setsansfont{SourceSansPro-Semibold.otf}

%%%% Set Page
\pagestyle{fancy}
\fancyhf{} 
\fancyfoot{}
\renewcommand{\headrulewidth}{0pt}
\renewcommand{\footrulewidth}{0pt}

%%%% Set URL Style
\urlstyle{same}

%%%% Set Indentation
\raggedbottom
\raggedright
\setlength{\tabcolsep}{0in}

%%%% Set Secondary Color
\definecolor{UI_blue}{RGB}{32, 64, 151}

%%%% Define New Commands
\usepackage[style=nature, maxbibnames=3]{biblatex}
\addbibresource{Publications.bib}

%%%% Bold Name in Publications
\renewcommand*{\mkbibnamegiven}[1]{%
\ifitemannotation{highlight}
{\textbf{#1}}
{#1}}

\renewcommand*{\mkbibnamefamily}[1]{%
\ifitemannotation{highlight}
{\textbf{#1}}
{#1}}

%%%% Set Sections formatting
\titleformat{\section}{
\color{UI_blue} \scshape \raggedright \large 
}{}{0em}{}[\vspace{-10pt} \hrulefill \vspace{-6pt}]

%%%% Set Subtext Formatting
\newcommand{\subtext}[1]{
#1\par\vspace{-0.2cm}}

% \newcommand{\subtextit}[1]{\vspace{0.15cm}
% \textit{ #1 \vspace{-0.2cm}} }

%%%% Set Item Spacing
\setlist[itemize]{align=parleft,left=0pt..1em}

%%%% New Itemize "Zitemize" Formatting - tighter spacing than itemize
\newenvironment{zitemize}{
\begin{itemize}\itemsep0pt \parskip0pt \parsep1pt}
{\end{itemize}\vspace{-0.5cm}}


%%%% Define Skills Bold Formatting
\newcommand{\hskills}[1]{
\textbf{\bfseries #1} }

%%%% Set Subsection formatting
\titleformat{\subsection}{\vspace{-0.1cm} 
\bfseries \fontsize{11pt}{2cm}}{}{0em}{}[\vspace{-0.2cm}]

%%%%%%% --------------------------------------------------------------------------------------
%%%%%%% --------------------------------------------------------------------------------------
%%%%%%%  END OF "DO NOT TOUCH" REGION
%%%%%%% --------------------------------------------------------------------------------------
%%%%%%% --------------------------------------------------------------------------------------

\begin{document}

%%%%%%% --------------------------------------------------------------------------------------
%%%%%%%  HEADER
%%%%%%% --------------------------------------------------------------------------------------
\begin{center}

    \begin{minipage}[b]{0.9\textwidth}
            \centering
            {\Huge Ruoqi Wang} \\ %
            \vspace{0.2cm}
			   Homepage: \href{https://wang-rq.github.io}{wang-rq.github.io} \space \space \space \space \space \space \space \space \space \space \space \space
Email: \href{mailto:wangrq29@mail2.sysu.edu.cn}{wangrq29@mail2.sysu.edu.cn} \space \space \space \space \space \space \space \space \space \space \space \space 
Tel: (+86) 132-0206-7930 
	\vspace{0.1cm}
    \end{minipage}% 

    
\vspace{-0.15cm} 

\end{center}




%%%%%%% --------------------------------------------------------------------------------------
%%%%%%%  EDUCATION
%%%%%%% --------------------------------------------------------------------------------------
\section{\large \textbf{Education}}

\subsection*{Sun Yat-sen University (SYSU),\space \space \space \space  B.Eng., \space \space \space \space    School of Computer Science and Engineering. \hfill Sep. 2018 --- Jun. 2022 } 
    \begin{zitemize}
        \item Major: Computer Science and Technology
		\item Weighted Average Grade: \textbf{89/100} (GPA: \textbf{3.8/4.0}) \\
		 - \space \space Freshman Year: 85/100, \space \space Sophomore Year: 91/100, \space \space Junior Year: 91/100
		\item Awards and Honors: \\
		 - \space \space Academic Excellence Scholarship, Sun Yat-sen University, 2020\\
		- \space \space Student Elite Representative, School of Computer Science and Engineering, Sun Yat-sen University, 2021\\
		

    \end{zitemize}
\vspace{0.2cm}



%%%%%%% --------------------------------------------------------------------------------------
%%%%%%%  PUBLICATIONS
%%%%%%% --------------------------------------------------------------------------------------
\section{\large \textbf{Research Papers}} 
\begin{enumerate}[itemindent=0em]

\item \textbf{Ruoqi Wang}, Ziwang Huang, Haitao Wang, Hejun Wu, ``AMMASurv: Asymmetrical Multi-Modal Attention for Accurate Survival Analysis with Whole Slide Images and Gene Expression Data, " \textit{IEEE International Conference on Bioinformatics and Biomedicine \textbf{(IEEE BIBM)}}, under review, preprint arXiv:2108.12565, 2021.

\item Ziwang Huang, Hua Chai, \textbf{Ruoqi Wang}, Haitao Wang, Yuedong Yang, Hejun Wu, ``Integration of patch features through self-supervised learning and transformer for survival analysis on whole slide images",  \textit{Medical Image Computing and Computer-Assisted Intervention Society \textbf{(MICCAI)}}, \textbf{accepted}, 2021.

\item Haitao Wang, Hejun Wu, \textbf{Ruoqi Wang}, Ziwang Huang, ``Deep Discriminative Feature Learning for Concrete Surface Damage Classification," \textit{IEEE Transactions on Industrial Informatics \textbf{(IEEE TII)}}, under review, 2021.

\item Haitao Wang, Yongqiang You, Hejun Wu, \textbf{Ruoqi Wang}, ``Discrete Contrastive Representation Learning for Reinforcement Learning," \textit{AAAI Conference on Artificial Intelligence \textbf{(AAAI)}}, under review, 2022.
\end{enumerate}




%%%%%%% --------------------------------------------------------------------------------------
%%%%%%%  ACADEMIC PROJECTS
%%%%%%% --------------------------------------------------------------------------------------
\section{\large \textbf{Research Projects}} %% (Or "Research", select as appropriate)

%%% ----- Best way to write items (Credit - FAANGPath)
        % \item Achieved X\% growth for XYZ using A, B, and C skills.
        % \item Led XYZ which led to X\% of improvement in ABC
        % \item Developed XYZ that did A, B, and C using X, Y, and Z. 


%%%%%%% ----------------------------------- Role 1 ----------------------------------- %%%%%%%
\centerline{\textcolor{UI_blue}{Artificial Intelligence in Healthcare}}
\vspace {6pt}
\subsection*{Asymmetrical multi-modal survival analysis using medical images and structured data.\hfill May. 2021 --- Aug. 2021} 
\subtext{Machine Perception Laboratory, SYSU }
\vspace {6pt}
My contributions:
    \begin{itemize}[topsep = 0 pt, itemsep = 0 pt, parsep = 1 pt]
       \item I independently designed an asymmetrical multi-modal attention mechanism (AMMA) to generate more flexible joint representation of medical images and structured data. 
		\item Different from previous works, AMMA can effectively utilize the intrinsic information within every modality and flexibly adapt to the modalities of different importance.
		\item I designed and conducted various experiments to verify the effectiveness of the new model.  On public datasets from TCGA, the results of the proposed method are 5\%-6\% higher (C-index) than other SOTA methods. 
        \item The article \textit{``AMMASurv: Asymmetrical Multi-Modal Attention for Accurate Survival Analysis with Whole Slide Images and Gene Expression Data"} was submitted to IEEE BIBM.
    \end{itemize}


\subsection*{Integration of Patch Features of Whole Slide Images through Self-Supervised Learning and Transformer for Survival Analysis. \hfill Dec. 2020 --- Mar. 2021}
\subtext{Machine Perception Laboratory, SYSU}  
\vspace {6pt}
My contributions:
    \begin{itemize}[topsep = 0 pt, itemsep = 0 pt, parsep = 1 pt]
        \item I conducted experiments on the influence of self-supervised learning for extracting the features of whole slide images (WSIs) and researched the effect of positional embedding of WSI patches. 
		\item The approach with self-supervised learning and position embedding outperformed the previous best approach by an average of 3\% (C-index) in survival prediction on three datasets.
		\item The article \textit{``Integration of Patch Features through Self-Supervised Learning and Transformer for Survival Analysis on Whole Slide Images"} was accepted by MICCAI 2021.
    \end{itemize}



\subsection*{The effect of surgery and drug treatment on the visual field progression of different glaucoma patients and glaucoma patients with comorbidities.\hfill Dec. 2020 --- Mar. 2021} 
\subtext{Zhongshan Ophthalmic Center, SYSU }
\vspace {6pt}
My contributions:
    \begin{itemize}[topsep = 0 pt, itemsep = 0 pt, parsep = 1 pt]
       \item I designed an efficient sample matching method based on the Levenshtein distance algorithm to solve the problems of inaccurate and incomplete information in the original medical dataset.  
		\item The proposed method increased the number of effective samples by 30\%, facilitating subsequent experiments.
		\item The results are expected to be published in 2022.

    \end{itemize}


\subsection*{Machine disease diagnosis on small and unbalanced datasets with multi-modal data. \hfill Nov. 2020 --- Dec. 2020} 
\subtext{Machine Perception Laboratory, SYSU} 
\vspace {6pt}
My contributions:
    \begin{itemize}[topsep = 0 pt, itemsep = 0 pt, parsep = 1 pt]
        \item I used multi-modal data (clinical data and omics data of patients with nasopharyngeal carcinoma) to get better joint representations for classification. 
		\item I proved the complementarity between data from two modalities.
		\item I studied the application of machine learning methods on small and uneven datasets. 
    \end{itemize}
\vspace {12pt}

\centerline{\textcolor{UI_blue}{Artificial Intelligence in Industry}}
\vspace {-10pt}
\subsection*{Deep discriminative feature learning on concrete surface images.\hfill Jul. 2021 --- Sep. 2021} 
\subtext{Machine Perception Laboratory, SYSU } 
\vspace {6pt}
My contributions:
    \begin{itemize}[topsep = 0 pt, itemsep = 0 pt, parsep = 1 pt]
		\item I participated in researching a novel end-to-end framework named Deep Discriminative Feature Learning (DDFL) based on Collective Matrix Factorization (CMF) and Vision Transformer (ViT) to extract and select discriminative features of crack images. 
		\item The learning framework integrates the deep feature learning and feature selection so that more
discriminative representation can be learned for crack classification.
		\item  Our method outperforms other mainstream classification models on four datasets in varieties of evaluation criteria.
		\item The article  \textit{``Deep Discriminative Feature Learning for Concrete Surface Damage Classification"} was submitted to IEEE TII.
    \end{itemize}

\vspace {6pt}

\centerline{\textcolor{UI_blue}{Smart City}}
\vspace {-12pt}

\subsection*{Smart balanced delivery task scheduling based on TSP solver. \hfill Jul. 2021 --- Aug. 2021} 
\subtext{Machine Perception Laboratory, SYSU } 
\vspace {6pt}
My contributions:
    \begin{itemize}[topsep = 0 pt, itemsep = 0 pt, parsep = 1 pt]
        \item I optimized one of the proposed algorithms by reducing loops and superimpose distance and time matrices, making the computational complexity become 1/2 of the original method. 
		\item I re-implemented a baseline method GCN-NPEC. 
		\item The results are expected to be published in 2021.
    \end{itemize}





%%%%%%% --------------------------------------------------------------------------------------
%%%%%%%  SKILLS
%%%%%%% --------------------------------------------------------------------------------------
\section{\large \textbf{Technical Skills}}
\begin{zitemize}

\vspace{0.2cm}

\item \textbf{Programming Languages:} \\ 
			Familiar \space with \space C/C++, \space Python, \space and \space MATLAB.

\vspace{0.2cm}

\item \textbf{Software Frameworks \& Tools:}\\
			 Familiar \space with \space PyTorch, \space CUDA, \space OpenMP, \space MPI, \space OpenCV, \space NumPy, \space Scikit-learn, \space Pandas,  \space LaTeX, \space and \space MySQL.


\end{zitemize}



%%%%%%% ---------------------------- END DOC HERE ---------------------------- %%%%%%% 
\end{document}