\documentclass[letterpaper,10pt]{article}

\usepackage{latexsym}
\usepackage[empty]{fullpage}
\usepackage{titlesec}
\usepackage{marvosym}
\usepackage[usenames,dvipsnames]{color}
\usepackage{verbatim}
\usepackage{enumitem}
\usepackage[hidelinks]{hyperref}
\usepackage{fancyhdr}
\usepackage[english]{babel}
\usepackage{tabularx}
\usepackage{multicol}
\usepackage{CJKutf8}
\input{glyphtounicode}
\usepackage{academicons}
\usepackage{color}
\usepackage{hyperref}

\hypersetup{
    colorlinks=true,
    linkcolor=blue,
    filecolor=blue,      
    urlcolor=blue,
    citecolor=cyan,
}

\usepackage[default]{sourcesanspro}
\usepackage[T1]{fontenc}

\pagestyle{fancy}
\fancyhf{} 
\fancyfoot{}
\renewcommand{\headrulewidth}{0pt}
\renewcommand{\footrulewidth}{0pt}


\addtolength{\oddsidemargin}{-0.5in}
\addtolength{\evensidemargin}{-0.5in}
\addtolength{\textwidth}{1in}
\addtolength{\topmargin}{-.5in}
\addtolength{\textheight}{1.0in}

\urlstyle{same}

\raggedbottom
\raggedright
\setlength{\tabcolsep}{0in}

\titleformat{\section}{
  \vspace{-4pt}\centering
}{}{0em}{}[\color{black}\titlerule\vspace{-5pt}]


\pdfgentounicode=1

\newcommand{\resumeItem}[1]{
  \item{
    {#1 \vspace{-2pt}}
  }
}

\newcommand{\resumeSubheading}[4]{
  \vspace{-2pt}\item
    \begin{tabular*}{0.97\textwidth}[t]{l@{\extracolsep{\fill}}r}
      \textbf{#1} & #2 \\
      \textit{#3} & \textit{ #4} \\
    \end{tabular*}\vspace{-7pt}
}

\newcommand{\resumeSubSubheading}[2]{
    \item
    \begin{tabular*}{0.97\textwidth}{l@{\extracolsep{\fill}}r}
      \textit{#1} & \textit{ #2} \\
    \end{tabular*}\vspace{-7pt}
}

\newcommand{\resumeProjectHeading}[2]{
    \item
    \begin{tabular*}{0.97\textwidth}{l@{\extracolsep{\fill}}r}
      #1 & #2 \\
    \end{tabular*}\vspace{-7pt}
}

\newcommand{\resumeSubItem}[1]{\resumeItem{#1}\vspace{-4pt}}

\renewcommand\labelitemii{$\vcenter{\hbox{\tiny$\bullet$}}$}

\newcommand{\resumeSubHeadingListStart}{\begin{itemize}[leftmargin=0.15in, label={}]}
\newcommand{\resumeSubHeadingListEnd}{\end{itemize}}
\newcommand{\resumeItemListStart}{\begin{itemize}}
\newcommand{\resumeItemListEnd}{\end{itemize}\vspace{-5pt}}

\begin{document}

\begin{CJK}{UTF8}{gbsn}

\begin{center}
    {\huge Ruoqi Wang} \\ \vspace{2pt}

    Ph.D. Candidate\\
    The Hong Kong University of Science and Technology (Guangzhou) \\
    No.1 Du Xue Rd, Nansha District, Guangzhou, China
    \begin{multicols}{2}
    \begin{flushleft}
    
    Email: \href{mailto:{rwang280@connect.hkust-gz.edu.cn}}{rwang280@connect.hkust-gz.edu.cn} \\
    Phone: \href{{your phone number}}{(+86)13202067930} \\
    URL: \href{{https://wang-rq.github.io}}{https://wang-rq.github.io} \\
    
    \end{flushleft}
    
    \begin{flushright}

    GitHub: \href{{https://github.com/wang-rq}}{https://github.com/wang-rq}\\
    Google Scholar: \href{{https://scholar.google.com/citations?user=AHk1W2EAAAAJ&hl=zh-CN}}{Ruoqi Wang} \\
    Orcid: \href{{https://orcid.org/0009-0005-3513-1945}}{0009-0005-3513-1945}
    
    
    \end{flushright}
    \end{multicols}
\end{center}

%-----------EDUCATION-----------
\section{\Large{Education}}

\textbf{The Hong Kong University of Science and Technology (Guangzhou),\space \space \space \space  Ph.D. Student. \space \space \space \space     \hfill Aug. 2022 -- Jun. 2026 (Expected)} 
\begin{itemize}
        \vspace{-2pt}
        \item \textbf{Program}: Ph.D. in Data Science and Analytics
        \vspace{-2pt}
        \item \textbf{GPA}: 3.9/4.0
    \end{itemize}

    
\textbf{Sun Yat-sen University,\space \space \space \space  B.Eng. \space \space \space \space     \hfill Sept. 2018 -- Jun. 2022 } 
\begin{itemize}
        \vspace{-2pt}
        \item \textbf{Major}: Computer Science and Technology
        \vspace{-2pt}
        \item \textbf{GPA}: 3.8/4.0
        \vspace{-2pt}
		\item \textbf{Awards and Honors}: \\
        \vspace{2pt}
		- \space \space Academic Excellence Scholarship, Sun Yat-sen University, 2020 \& 2021\\
        \vspace{2pt}
		- \space \space Student Elite Representative, School of Computer Science and Engineering, Sun Yat-sen University, 2021\\
        
		- \space \space Excellent Undergraduate Thesis (\textbf{rank 1/444}), Sun Yat-sen University, 2022\\
    \end{itemize}



%-----------Research Topics-----------
% \section{\Large{Research Interests}}
% \begin{itemize}
% \item {\textbf{AI for Astronomy \& Scientific Imaging}:} Radio interferometric imaging/reconstruction, generative models, inverse problem, application of large models.
% \item {\textbf{Vision–Language models \& Multi-Modal Learning}:} Especially for scientific multi-modal data.
% \item {\textbf{Robust \& Generalizable Representation Learning}:  Domain adaptation, domain generalization, data augmentations. }
% \item {\textbf{Semi-/Self-Supervised Learning}: Use of unlabeled or sparsely labeled scientific data. }
% \item {\textbf{AI for Healthcare}: Medical image analysis, multi-modal learning for healthcare. }
% \end{itemize}

\section{\Large{Research Interests}}

\textit{Theme: Trustworthy \& efficient machine learning for scientific and real-world application.}

\begin{itemize}
\item \textbf{Scientific \& Physics-Guided ML}: inverse problems; data reconstruction; physical priors; computational imaging
\item \textbf{Foundation \& Multi-modal Models}: vision–language models; cross-domain alignment; scientific foundation models
\item \textbf{Reliable \& Efficient Learning}: robustness/generalization; semi-/self-/weak supervision; domain adaptation
\item \textbf{AI Applications}: astronomy, healthcare and industry
\end{itemize}

%-----------PUBLICATIONS-----------
\section{\Large{Publications}}

\textbf{Conference Papers:}
 \begin{itemize}
    \item{
    \underline{Ruoqi Wang}, Haitao Wang, Qiong Luo, "{GalaxAlign: Mimicking Citizen Scientists' Multimodal Guidance for Galaxy Morphology Analysis}", arXiv preprint arxiv.org/abs/2411.19475, accepted by 33rd ACM International Conference on Multimedia (ACM MM 2025).
    }
    
    \item{
    \underline{Ruoqi Wang}, Haitao Wang, Qiong Luo, Feng Wang, Hejun Wu, "{VisRec: A Semi-Supervised Approach to Radio Interferometric Data Reconstruction}", Proceedings of the AAAI Conference on Artificial Intelligence. Vol. 39. No. 1. 2025.
    }
    
    \item{
    \underline{Ruoqi Wang}, Zhuoyang Chen, Jiayi Zhu, Qiong Luo, Feng Wang, "{PolarRec: Improving Radio Interferometric Data Reconstruction Using Polar Coordinates}", The IEEE/CVF Conference on Computer Vision and Pattern Recognition (CVPR), pages 12841-12850, 2024. 
    }

    \item{
    \underline{Ruoqi Wang}, Zhuoyang Chen, Qiong Luo, Feng Wang, "{A Conditional Denoising Diffusion Probabilistic Model for Radio Interferometric Image Reconstruction}", 26th European Conference on Artificial Intelligence (ECAI), pages 2499 - 2506, 2023.
    }

    \item{
    \underline{Ruoqi Wang}, Ziwang Huang, Haitao Wang, Hejun Wu, "{AMMASurv: Asymmetrical Multi-Modal Attention for Accurate Survival Analysis with Whole Slide Images and Gene Expression Data}", IEEE International Conference on Bioinformatics and Biomedicine (BIBM), pages 757-760, 2021.
    }

    \item{
    Ziwang Huang, Hua Chai, \underline{Ruoqi Wang}, Haitao Wang, Yuedong Yang and Hejun Wu, "{Integration of Patch Features through Self-Supervised Learning and Transformer for Survival Analysis on Whole Slide Images}", International Conference on Medical Image Computing and Computer-Assisted Intervention (MICCAI), pages 561–570, 2021.
    }

 \end{itemize}


%-----------Preprints-----------
\vspace{2pt}
\textbf{Ongoing Papers:}
\begin{itemize}
    \item{
    \underline{Ruoqi Wang}, Haitao Wang, Shaojie Guo, Qiong Luo, "{Improving Out-of-Domain Robustness with Targeted Augmentation in Frequency and Pixel Spaces}", arXiv preprint arxiv.org/abs/2505.12317, 2025.
    }

    \item{
    Zhuoyang Chen, \underline{Ruoqi Wang}, Qiong Luo, "{ProtAug: Utilizing Self-Supervised Protein Language Models for Effective Protein Sequence Augmentation}", 2025.
    }
\end{itemize}


%-----------Research Experience-----------
\section{\Large{Research Experience}}

% \textit{\textbf{A.} Ph.D. Phase (2022 - Present)}
% \begin{itemize}
%     \item \textbf{Thesis Topic: } Machine Learning for Astronomical Data Reconstruction and Analysis
%     \item \textbf{Description: } Proposed a machine-learning-based pipeline for radio-interferometric imaging and analysis: \\
%     - Imaging: VIC-DDPM (spectral+spatial diffusion), PolarRec (polar transformer), VisRec (semi-supervised). \\

%     - Analysis: GalaxAlign for morphology classification \& retrieval; domain adaptation across telescopes. \\

%     - Outcome: An integrated pipeline containing low-level reconstruction and high-level analysis, enabling scalable and robust galaxy morphology studies.
% \end{itemize}

% \textit{\textbf{A.} Undergraduate Phase (2020 - 2022)}
% \begin{itemize}
%     \item \textbf{Thesis Topic: } Machine-Learning-Based Survival Analysis on Multi-Modal Medical Data
%     \item \textbf{Description: } Proposed an multimodal survival model with WSI-guided attention over gene expression: \\
%     - Problem: Prior WSI+gene survival models ignore whole-slide context, assume equal modality importance. \\
%     - Method: AMMASurv with AMMA—heterogeneous directed graphs where WSI induces gene-expression features (asymmetric flow).\\
%     - Result: Better performance robust multi-modal survival prediction.
    
% \end{itemize}

\textbf{Ph.D. Phase} \hfill 2022--Present \\
\emph{Topic: Machine Learning for Astronomical Data Reconstruction and Analysis}
\begin{itemize}
  \item \textbf{Problem:} Sparse/noisy visibilities, artifact-prone imaging, label scarcity, limited robustness, and poor generalization.
  \item \textbf{Method:} \textbf{VIC-DDPM} (spectral+spatial diffusion), \textbf{PolarRec} (polar transformer), \textbf{VisRec} (semi-supervised); \textbf{GalaxAlign} for morphology classification \& retrieval; \textbf{Pixel-Frequency Connect} for domain adaptation across different measuring instruments.
  \item \textbf{Impact:} End-to-end pipeline from low-level reconstruction to high-level analysis; scalable and robust galaxy morphology studies.
\end{itemize}

\textbf{Undergraduate Phase} \hfill 2020--2022 \\
\emph{Topic: Machine-Learning–Based Survival Analysis on Multi-modal Medical Data}
\begin{itemize}
  \item \textbf{Problem:} Prior WSI + gene expression survival models ignore whole-slide context, assume equal modality importance, and are sensitive to noisy gene expression.
  \item \textbf{Method:} \textbf{AMMASurv} with \textbf{AMMA}---heterogeneous directed graphs; WSI-guided attention induces gene-expression features (asymmetric information flow).
  \item \textbf{Impact:} Improved multi-modal survival prediction on two public cancer datasets.
\end{itemize}



%-----------Teaching Experience-----------
\section{\Large{Teaching \& Mentoring}}
\begin{itemize}
    \item  Teaching Assistant:  \\
    - \textit{Deep Learning in Data Science}, The Hong Kong University of Science and Technology (Guangzhou).   \hfill Spring 2025 \\
    \vspace{2pt}
    - \textit{Physical Education — Tennis}, The Hong Kong University of Science and Technology (Guangzhou). \hfill Fall 2024
    \item  Invited Lecturer: \\
    - \textit{Artificial Intelligence Practice 2025}, Sun Yat-sen University. \hfill Summer 2025
    
\end{itemize}



%-----------PROGRAMMING SKILLS-----------
\section{\Large{Skills}}
 \begin{itemize}
    \item{
     \textbf{Programming}{: Python; C/C++; MATLAB; JavaScript.} 
     }
     \item{
     \textbf{Libraries/Tools}{: Includes PyTorch; OpenCV; NumPy; Scipy; Torchvision; Pandas; Scikit-learn; Matplotlib; Seaborn.}
     }\\
     \item{\textbf{Systems/HPC}{: CUDA; multi-GPU (DDP/Deepspeed); Slurm; Docker/conda; OpenMP, Open MPI.\\}}
     \item{\textbf{Astro}: Astropy; radio interferometry simulators; FFT pipelines; FITS/HDF5.}
     \item{
     \textbf{Languages}{: Mandarin (native); English (fluent)}
     }
     \item{
     \textbf{Hobbies}{: Tennis; Reading; Traveling}
     }
 \end{itemize}

%-----------PROGRAMMING SKILLS-----------
\section{\Large{Service}}
\begin{itemize}
    \item Conference Reviewer: CVPR 2024-2025, ICCV 2025, ICML 2025, NeurIPS 2024-2025, ICLR 2025-2026, AAAI 2024-2025, ACM MM 2025.
    \item Journal Reviewer: Publications of the Astronomical Society of Australia.
\end{itemize}

\section{\Large{References}}
\begin{itemize}
    \item \textbf{Prof. Dr. Qiong Luo},  Email: \href{mailto:{luo@ust.hk}}{luo@ust.hk}\\
    Ph.D. advisor, Department of Computer Science and Engineering, The Hong Kong University of Science and Technology 
    
    \item \textbf{Prof. Dr. Hejun Wu}, Email: \href{mailto:{wuhejun@mail.sysu.edu.cn}}{wuhejun@mail.sysu.edu.cn}  \\
    Undergraduate Advisor, School of Computer Science and Engineering, Sun Yat-sen University \\
    

    \item \textbf{Prof. Dr. Feng Wang}, Email: \href{mailto:{fengwang@gzhu.edu.cn}}{fengwang@gzhu.edu.cn} \\
    Collaborator, Center for Astrophysics, Guangzhou University \\
    
\end{itemize}


% %-----------Research Topics-----------
% \vspace{2pt}
% \section{\Large{Research Topics}}

% \textbf{Astronomical data reconstruction:} Sparse to dense reconstruction of astronomical data.

% \textbf{Galaxy Image Recovery:} Removing noise and artifacts in galaxy images.



% %-----------PUBLICATIONS-----------
% \vspace{2pt}
% \section{\Large{Publications}}
%  \begin{itemize}[leftmargin=0.15in, label={}]
%     \item{
%     \underline{Ruoqi Wang}, Zhuoyang Chen, Jiayi Zhu, Qiong Luo, Feng Wang, "\textbf{PolarRec: Radio Interferometric Data Reconstruction with Polar Coordinate Representation}", The IEEE/CVF Conference on Computer Vision and Pattern Recognition (CVPR), 2024, accepted. 
%     }

%     \item{
%     \underline{Ruoqi Wang}, Zhuoyang Chen, Qiong Luo, Feng Wang, "\textbf{A Conditional Denoising Diffusion Probabilistic Model for Radio Interferometric Image Reconstruction}", 26th European Conference on Artificial Intelligence (ECAI), pages 2499 - 2506, 2023.
%     }

%     \item{
%     \underline{Ruoqi Wang}, Ziwang Huang, Haitao Wang, Hejun Wu, "\textbf{AMMASurv: Asymmetrical Multi-Modal Attention for Accurate Survival Analysis with Whole Slide Images and Gene Expression Data}", IEEE International Conference on Bioinformatics and Biomedicine (BIBM), pages 757-760, 2021.
%     }

%     \item{
%     Ziwang Huang, Hua Chai, \underline{Ruoqi Wang}, Haitao Wang, Yuedong Yang and Hejun Wu, "\textbf{Integration of Patch Features through Self-Supervised Learning and Transformer for Survival Analysis on Whole Slide Images}", International Conference on Medical Image Computing and Computer-Assisted Intervention (MICCAI), pages 561–570, 2021
%     }

%  \end{itemize}


% %-----------Preprints-----------
% \vspace{2pt}
% \section{\Large{Preprints}}
%  \begin{itemize}[leftmargin=0.15in, label={}]
%     \item{
%     \underline{Ruoqi Wang}, Haitao Wang, Qiong Luo, Feng Wang, Hejun Wu, "\textbf{VisRec: A Semi-Supervised Approach to Radio Interferometric Data Reconstruction}", arXiv preprint arXiv:2403.00897, 2024. 
%     }
% \end{itemize}


% %-----------PROGRAMMING SKILLS-----------
% \vspace{2pt}
% \section{\Large{Skills and Interests}}
%  \begin{itemize}[leftmargin=0.15in, label={}]
%     \item{
%      \textbf{Programming Languages}{: C/C++, Python, MATLAB} \\
%      \textbf{Libraries/Tools}{: OpenCV, PyTorch, NumPy, Scikit-learn, Pandas, LaTeX}\\
%      \textbf{Languages}{: Mandarin (native), English (fluent)} \\
%      \textbf{Interests}{: Tennis and Traveling}
     
%     }
%  \end{itemize}


\end{CJK}

\end{document}
